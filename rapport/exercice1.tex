%% LyX 2.3.3 created this file.  For more info, see http://www.lyx.org/.
%% Do not edit unless you really know what you are doing.
\documentclass[a4paper,english]{article}
\usepackage[T1]{fontenc}
\usepackage[latin9]{inputenc}
\usepackage{amsmath}

\makeatletter

%%%%%%%%%%%%%%%%%%%%%%%%%%%%%% LyX specific LaTeX commands.
\pdfpageheight\paperheight
\pdfpagewidth\paperwidth


\makeatother

\usepackage{babel}
\begin{document}

\section*{Exercice 1}

Soit $m$ un entier de $n$ chiffres en base 10. En stockant les chiffres
de $m$ en base $10$ dans un tableau de byte, le nombre de bits utilis�s
est $q=8n$, car chaque ``case'' du tableau utilise $8$ bits. Calculons
le nombre $q'$ de bits utilis�s si nous pouvions stocker l'�criture
binaire de $m$ dans la m�moire :

\begin{align*}
m\text{ utilises }q'\text{ bits} & \Longleftrightarrow2^{q'-1}\leq m<2^{q'}\\
 & \Longleftrightarrow q'-1\leq\frac{\ln m}{\ln2}<q'\\
 & \Longleftrightarrow\frac{\ln m}{\ln2}<q'\leq\frac{\ln m}{\ln2}+1
\end{align*}
De plus 
\begin{align*}
m\text{ a \ensuremath{n} chiffres en base 10} & \Longleftrightarrow10^{n-1}\leq m<10^{n}\\
 & \Longleftrightarrow(n-1)\frac{\ln10}{\ln2}\leq\frac{\ln m}{\ln2}<n\frac{\ln10}{\ln2}
\end{align*}
Donc
\[
(n-1)\frac{\ln10}{\ln2}<q'<n\frac{\ln10}{\ln2}+1
\]
Posons $e=\left|\frac{q'-q}{q}\right|$. Alors 
\begin{align*}
 & \left|1-\frac{n\frac{\ln10}{\ln2}-1}{8n}\right|<e<\left|1-\frac{(n-1)\frac{\ln10}{\ln2}}{8n}\right|\\
\Longleftrightarrow & \left|1-\frac{\ln10}{8\ln2}+\frac{1}{8n}\right|<e<\left|1-\frac{\ln10}{8\ln2}+\frac{\ln10}{8n\ln2}\right|
\end{align*}
Donc � la limite on a
\[
e=1-\frac{\ln10}{8\ln2}\approx0.5848
\]
On voit alors que en stockant les chiffres de $m$ en base $10$ dans
un tableau de byte, $58\%$ de la m�moire est utilis�e inutilement
par rapport � l'�criture binaire classique.
\end{document}
